\section{Introduction}

In the previous chapter we found, through historical survey and qualitative discussion, that dispersive measurement allows multiplexed qubit measurement while partially preserving the qubit coherence.
In this chapter, we analyze dispersive measurement in full quantitative detail.

In dispersive measurement, where a qubit is coupled off resonance to a linear resonator, the resonator's frequency depends on the qubit's quantum state.
Photons populating the resonator acquire a phase shift which depends on the resonator's frequency and therefore on the qubit state.
In other words, the photons are ``dispersed'' in a way which depends on the qubit state.
Therefore, the qubit state is measured by probing the resonator and measuring the phase of the outgoing photons.
The analysis comes naturally in two steps.
First, we develop the Hamiltonian for a qubit coupled to a resonator with large qubit-resonator detuning.
From the Hamiltonian we find an equation expressing the resonator frequency shift in terms of other parameters in the system, such as the qubit-resonator coupling strength and detuning.
Second, we analyze the classical problem of measuring the resonator's resonance frequency through microwave scattering.
Combined, these analyses show how the scattered microwave signal carries the information of the qubit state.
We then describe the process by which the qubit state collapses as information is carried away by the dispersed photons.
At the end, we present additional details of the dispersive measurement circuit which come into play in a practical lab setting where amplifier saturation is an important limitation.
