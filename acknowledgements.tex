\begin{acknowledgements}

I arrived at UCSB with little lab experience: I didn't know how to use vacuum, cryogenics or RF equipment, and I barely knew how to program a computer.
In pursuing the work described in this thesis, I learned, through the example set by other lab members, not just these technical skills, but how to work in a spirit of collaboration and shared success.
I think this overwhelmingly positive lab culture is the most valuable thing I take away from my experience here and for that I thank you all.

I would like to especially acknowledge Professor John Martinis.
John's strategy that ``everyone learns to do everything'' has provided me with a diverse scientific toolbox which I will carry and use for the rest of my life.
His often repeated demand ``show me data'' has strongly shaped how I approach problems both in research and outside the lab.
Doing physics with John has been a lot of fun.
I had the pleasure of working directly with him on, among other things, the first generation of ADC boards, a project which was fun and exciting and brought a whole new set of possibilities to the lab, including the work described in this thesis.
John encouraged my inclination to understand things at a deep level.
Even after probably the third time I asked him to go for lunch to talk about quantum noise he never expressed frustration or boredom.
Instead, he said one of the most gratifying things a grad student could ever hear: ``I'm so glad you ask these kinds of questions''.
Through John I've had unusual and wonderful opportunities to travel: Zhejiang, and the Les Houches summer school formed two of my most coveted memories.
John, you've also been a great friend.
The road trip to see the eclipse at Zion, petanque and hiking at Les Houches, volleyball, and your unwavering support during a few personal road bumps are just a small fraction of the ways you helped me succeed.

I would like to acknowledge the other members of my thesis committee: professors Chetan Nayak and Andrew Cleland for reading my thesis and for their kind and positive feedback after the defence talk.

When I first arrived at UCSB, Markus Ansmann met me with kindness and encouraged my curious nature.
At the very beginning, I noticed that Markus didn't just make his own experiments work; in taking the time and effort to produce the right tools for the job, he made everyone's experiments work.
This is no more apparent than in LabRAD, the software backbone of the Martinis lab which now runs in at least fifteen other labs across the globe.
Markus, your kindness and your approach to research, emphasizing contributions to the team as a whole, set my expectations for how life in the lab should go, and for that I am grateful.

Radek Bialczak taught me how to operate the refrigerator (i.e. how to avoid making pipe bombs out of nitrogen dewars) and tutored me in qubit physics as I began my first project measuring noise, an extension of one of Radek's own projects.
Radek, your work ethic, enthusiasm for passing on knowledge, and ability to tackle completely new projects inspired me.
You mention my ``true passion for physics'' in your thesis acknowledgements; I hope you know that yours helped me pay attention to my own.

The day I realized Erik Lucero designed and built the DC control electronics was the day I really understood what it meant to ``get things done'' and take pride in your work.
Erik taught me how to use the cleanroom, swear like a sailor, and make physics cool.
He also totally saved my butt.
My first project, and first trip into the cleanroom, required the seven layer phase qubit process.
When Erik found this out, he traded fabs with me, giving me a more manageable three layer process from which to learn.
Thanks for looking out for me, Erik.
It was a true pleasure pumping tunes in the DR lab, eating wings at Jax, and taking the ``sorcery'' to Zhejiang (what \emph{really} happened that afternoon?).
Your dedication to keeping in touch with old colleagues and friends warms the heart, and assures me that we'll be collaborating on something delicious in the near future.

Whatever sorcery skills I now possess came through Matthew Neeley.
In my early days at UCSB, we only had one DR.
I, as the new guy, had to take the graveyard shift to get data.
With nobody around to help, I wasted many nights with software bugs that I didn't know how to fix myself.
Those painful experiences left me with a strong desire to be self-sufficient, and so I naturally gravitated towards Matthew.
No problem held Matthew up for long, least of all with software.
I admired Matthew's ability to move from theory to experiment, from software to electronics.
His never-ending patience in explaining what a future is and why I should care allowed me to carry the lab's software torch after he and Markus graduated.
Thanks for your patience as I learned the zen of lab software, your always positive attitude, and your genuine interest in brain teasers.
I'm looking forward to working together in the future.

Throughout my Ph.D., my conversations with Professor Robert McDermott on physics, and his kind words regarding my own future have been a source of confidence.
Watching Robert's research group in Wisconsin slowly but surely crack the flux noise problem has been absolutely inspirational.
Thanks, Robert, for your encouragement and for showing by example the value of really hard work.

From Max Hofheinz I learned two invaluable lessons.
First, teamwork is fun, and second, don't try reading articles right after lunch.
Thanks, Max, for getting me started taking qubit data and for your constant good will.

Haohua Wang was the first person I can remember ever asking me for an opinion on a physics problem.
As a young grad student, that came with a valuable confidence boost.
This was reinforced several years later when Haohua invited Erik and myself to visit his lab in Zhejiang for invited talks and to help set up the lab.
That was a truly memorable trip; Haohua treated us like royalty and the evidence is in Erik's photographs of the food (he shot every meal).
Haohua, you are a great friend and a role model of a hard working physicist.
You went out of your way to help me out when I was down and I want you to know that I truly appreciate it.

Tsuyoshi Yamamoto came to us from Japan for a year to learn our lab techniques.
He may have gone home with some UCSB sorcery, but he left us with a controlled-Z gate, a pretty good trade for the UCSB group.
Tsuyoshi, I hope I can say that I absorbed some of your infinite selflessness.

Charles Neill is for me the ``other'' guy in the lab who worked through the basic mathematics of qubit coupling and control.
The best way to understand something is to work it out for yourself, and so I was very happy to be able to compare notes with Charles.
Charles, your willingness to talk and understand the details been a pleasure for me.

There are several people with whom I had only passing interaction in the lab, but whose contributions to the lab environment and infrastructure were essential to my own success.
Aaron O'Connell's success turning a ridiculously complex (thirteen steps?) fabrication process into the world's first ground state mechanical oscillator, gave me an appreciation for the fab development process which I had previously not understood.
Yi Yin once simply told me ``that's not a reason to give up'', words I will never forget.
Matteo Mariantoni offered many encouraging words and advice on research and graduate student survival skills.
He also catalysed our interaction with Austin Fowler, an interaction which laid the motivation for my work on fast measurement.
IoChun Hoi took an interest to our filter system and provided me beta testing for explanations I would later use in public.
Pedram Roushan lent valuable perspective and insight into the landscape of academic research.
Thanks also for the very special Persian sweets.
Jimmy Chen, Andrew Dunsworth, Chris Quintana, Brooks Campbell, and Ben Chiaro are carrying on the Martinis lab culture and taking superconducting qubits to the next level.
Our discussion of professional kindness on the drive back from CSQ San Francisco gave me great confidence that you will do it in high style.
Chris, thank you for reading my thesis and giving good comments.
I learned some new things about quantum noise (again) in addressing them.

Now I come to the five gentlemen with whom I had the distinct pleasure of sharing an office: Julian Kelly, Tony Megrant, Peter O'Malley, Ted White, and Amit Vainsencher.
First of all, I'd like to acknowledge your willingness to make the move on the office when and how we did.
It was a dangerous mission, but I think it paid off.

Peter took up responsibility for the Qubit Sequencer, a mission critical layer in our qubit control stack, and maintained it as our needs evolved.
This was a major weight off my shoulders after Matthew graduated, and for that I'm very grateful.
Peter is also an experienced zymologist, a jazz connoisseur, and a compelling debater.
Peter, I have fond memories of working on lab infrastructure, strawberry wine, protracted philosophical discussions, stinky tea, and grumping at each other like old men.
Also, Peter, I'm truly sorry you had to take the heat for catfax.

In terms of research style, Julian and I are in many ways complimentary.
The speed with which he works in the cleanroom and at the keyboard makes me uncomfortable, plain and simple, but I think this is the heart of a really productive collaboration.
Julian's full-speed-ahead style pushed me to get the \emph{gateCompiler} working faster than I would have on my own, and I think we can both look proudly at how much that little program has improved quality of life in the DR room.
Julian took pity on my fading memory of qubit fab and buddied with me to get the fast measurement and five Xmon wafers done.
Julian, I thank you or your hard work on the Xmon, the qubit which enables us to take everything to the next level.
Your enthusiasm is a hallmark of the UCSB lab and something I truly try to make part of myself.
Don't forget Gatorade.

Tony Megrant's work with epitaxial aluminum films lead to the long qubit $T_1$'s the group now enjoys.
These long $T_1$'s were a critical element in achieving the measurement accuracy described in this thesis.
Tony, I look at your ability to juggle multiple projects as a quality to strive for, and I've enjoyed talking with you about fab ideas and life beyond the physics lab.

It's been said that I'm an inductor, with voltage spikes at any change in current.
If that's true, then Ted White is the complimentary capacitor: nothing phases Ted.
We make a resonant combination, so it's no surprise then, that between his (and Josh's) paramp and my (and Evan's) filter, we were able to get such good qubit measurement.
Seriously Ted, I've had a great time working with you, and I really appreciate your willingness to explain parametric amplification and quantum noise to me, no matter how many tries it takes.

James Wenner was my comrade in arms during a several month long benchmarking project.
I would not have finished that project on time without him.
Jim, I appreciate the way you take care of the lab family (e.g. taking over the snack fund), and your pleasant farewells each evening.
Your ability to remain polite in any situation is unique and much appreciated.

I lost count of the number of times Amit Vainsencher fixed something for me in the lab.
If something doesn't work, you ask Amit, and even if he's never seen your circuit board before, he'll find the solder bead which is shorting it out.
When you need to talk something out, Amit's the guy you want listening.
His genuine interest and pertinent questions will lead you to a solution.
I discovered this though many trips with Amit to Noodle City for Vietnamese noodles, a tradition which lifted my spirits and got me out of lab several times when I might otherwise have sat stuck in a rut all night.
Amit, your methodical, patient approach to research, and the humble attitude you keep despite your obvious talent and accomplishments is a standard by which I try to calibrate myself.
Thanks for the noodles, the contraband radio station, chocolate milk, peanut butter pretzels, and of course the DR cooldown Nintendo which made those twelve hour days of watchful waiting a lot more fun.
We still need to finish Mario 2.
wut?

I made the switch to working on state measurement pretty late in my grad school career.
Diving into resonator scattering, analog-to-digital conversion electronics, and dispersive qubit-resonator coupling physics, I felt in over my head.
The other two big qubit related projects, gates and amplifiers, had teams of two: one grad student and one post-doc, but I was by myself and concerned that there was more work ahead than I could handle.
In walked Evan Jeffrey.
One afternoon, I was hunched over one of Erik's fastbias cards and accompanying circuit diagram when Evan walked up behind me.
He asked what I was doing, and then when I explained that I was looking for the dominant source of low frequency noise, he paused for maybe ten seconds and then pointed to an element on the diagram.
He was right.
An opportunity had presented itself so I asked if he might just be interested in fast state measurement.
I'm very glad I did.
Not only did we knock out some pretty cool experiments, but it turned out Evan was really excited about LabRAD so I finally had a software buddy.
Evan is not only amazingly technically proficient, he's hilarious, kind, and makes excellent beer and smoked meats.
Evan, your companionship in the lab has been great.
I've really enjoyed solving problems (paramp gain modulator) with you.
Thanks for the beer, the brisket and ribs, the scissor spring boxing glove, the four-channel python debugger, your endless patience explaining new coding ideas, and your friendship.

Dr. Yu Chen built the first qubit sample at UCSB to use RF based measurement.
That project opened the door for dispersive readout at UCSB, eventually leading to the bandpass filter experiments.
I would like to thank you, Yu, for sharing your successes, spicy soup, baozi, and always reminding me of the importance of life outside the lab.
I may have a Ph.D. now, but you're always ``Dr. Chen'' to me.

Rami Barends's hard work with Julian on the Xmon qubit enabled the entire group to move forward with more complex and interesting experiments.
Without the Xmon there would have been no fast measurement.

Similarly, Josh Mutus's work (with Ted) on the high dynamic range, high bandwidth parametric amplifier played a critical role in the success of the work described in this thesis (see chapter \ref{ch:ExperimentalSetup}).
Josh joined me in regular therapeutic sessions of complaining about anything and everything under the sun, and made me dinner more than a few times.
It's been fun working in the lab, struggling to understand new physics together, and pushing our software boundaries to bigger and better things.

Austin Fowler's work on the surface code theory was a major factor in the direction of the lab.
His thorough study of the threshold requirements on state measurement gave Evan and I a concrete goal, and thus invaluable motivation.
Austin also picked up the GHz FPGA boards torch while I was writing this thesis, a real feat in itself and even more so given that Austin calls himself a theorist.
I have thoroughly enjoyed learning about error correction from you, Austin, and I hope we can continue what we've started.

Professor Alexander Korotkov has been a constant source of truly exquisite theoretical insight on topics ranging from fundamentals of parametric amplification to measurement induced dephasing (see chapter \ref{ch:DispersiveMeasurement}).
I remember him explaining how phase sensitive amplifiers achieve lower noise on a napkin during lunch at the Arbor.
His talks at the Les Houches summer school opened my eyes to a new way of understanding quantum measurement, and inspired much discussion amongst the students.
Sasha, I savour your visits to Santa Barbara.
Whether we're talking physics, philosophy, or even ``kindergarten'' logic puzzles it's always stimulating and a very good time.

Professor Andrew Cleland provided extensive feedback on the thesis itself.
I have also benefited on several occasions from Andrew's willingness to discuss details of difficult physics problems.
While he and the others were writing the surface code paper, I kept going to Andrew's office demanding more details on how the grid of physical qubits really encodes a logical qubit.
Andrew received my curiosity gratefully and responded with earnest devotion to improving the paper's pedagogical qualities.
I really appreciated that, Andrew.
I also want to thank you for giving such clear and thorough physics talks; I take them as models for my own.

Michel Devoret's course in fluctuations and noise was my first introduction to quantum devices.
So, for me, Michel is responsible for this whole business.
I was drawn to Michel's deep devotion to physics pedagogy and the cohesiveness with which he painted the intertwining web of ideas in physics.
During the Les Houches school, Michel offered me the most succinct and positive encouragement I was was given as a graduate student.
This came at a time that I needed it.
Thank you, Michel, for your devotion to your students, your contributions to the field, and for helping a struggling grad student feel good about himself.

To Jason, Evelyn, Casey, Jake and Jessa, I thank you for your endless hospitality and friendship.
In those first few years you guys made Santa Barbara start to feel like a home.
Jason, the fact that I could complete at Ph.D. in physics but still can't beat you in Smash Brothers is... something.

Faye, with your honesty and candor you've helped me own my nature, flaws included.
Thank you.

Mom and Dad, thank you so much for visiting me and taking family summer vacations while I was out here.
I know I took too long to unwind each time, but that just shows how much I needed you.
Thanks for teaching me to think, to cook, to try really hard and play for the long game.
Your example is a constant standard for me in all things.

It's one thing to get a Ph.D., but another entirely to do it with a smile.
So many of those smiles came from Anna Nierenberg.
Anna, Homey, you showed me the value of going to sleep, eating regularly, and just being happy.
You know, like an octopus.
Having you in my daily life inspired me to be who I want.
Thank you for listening to me whenever I needed it.
Our trips to San Diego, Monterey, Jackson, Les Houches, Hawaii, and Yosemite are my best memories of the last few years.

\end{acknowledgements}
